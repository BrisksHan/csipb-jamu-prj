%% This is an example first chapter.  You should put chapter/appendix that you
%% write into a separate file, and add a line \include{yourfilename} to
%% main.tex, where `yourfilename.tex' is the name of the chapter/appendix file.
%% You can process specific files by typing their names in at the 
%% \files=
%% prompt when you run the file main.tex through LaTeX.
\chapter{Metode Prediksi} \label{chap:prediksi}

Pada bab ini akan dibahas sedikit mengenai metode prediksi yang digunakan pada \emph{predictor} Ijah Webserver.

\section{BLM-NII}
BLM-NII adalah metode prediksi yang merupakan pengembangan dari metode BLM \emph{(Bipartite Local Model)}. BLM sendiri sudah terbukti efektif dalam prediksi interaksi target-obat, namun karena bersifat \emph{local model}, terdapat kelemahan yaitu tidak bisa memprediksi obat baru atau target baru yang belum pernah diketahui interaksinya sebelumnya. Oleh karena itu BLM dikembangkan dengan menambahkan prosedur NII \emph{(Neighbor-based Interaction-profile Inferring)} untuk menangani prediksi kandidat obat baru atau target baru tersebut. Secara spesifik, profil interaksi yang diinferensi akan diperlakukan sebagai label informasi dan digunakan sebagai model pembelajaran untuk mengenali kandidat tersebut. Kemampuan ini penting untuk mencari target dari kandidat obat baru dan sebaliknya, mencari obat dari kandidat target baru. Metode NII menurunkan model data latih dari tetangga \emph{(neighbor)} dari kandidat yang akan diinferensi. Definisi \emph{neighbor} pada konteks ini yaitu entitas lain (drug atau target) yang memiliki kemiripan tinggi dengan entitas kandidat. Kemiripan ditentukan dari struktur kimia.

\section{WNN-GIP}
WNN-GIP adalah metode prediksi yang ditujukan untuk memprediksi interaksi senyawa dan obat baru yang belum diketahui interaksinya. Metode ini menggunakan interaksi obat yang sudah diketahui sebelumnya untuk memprediksi yang baru dengan menggunakan algoritma \emph{regularized least square algorithm} yang menggabungkan kernel yang dikonstruksi dari profil interaksi senyawa obat dengan target. Profil interaksi ini dibangun dengan algoritme WNN \emph{(Weighted Nearest Neighbor)} menggunakan informasi kimia dan informasi interaksi yang sudah diketahui dalam dataset. WNN sendiri dapat digunakan sendiri secara \emph{standalone} untuk memprediksi interaksi obat baru. Selain digunakan secara \emph{standalone}, metode ini juga dapat digabungkan dengan metode GIP. Metode-metode ini dapat disesuaikan juga untuk memprediksi interaksi target baru atau bahkan keduanya (obat daru dan target baru).