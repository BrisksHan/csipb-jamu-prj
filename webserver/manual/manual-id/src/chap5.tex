%% This is an example first chapter.  You should put chapter/appendix that you
%% write into a separate file, and add a line \include{yourfilename} to
%% main.tex, where `yourfilename.tex' is the name of the chapter/appendix file.
%% You can process specific files by typing their names in at the 
%% \files=
%% prompt when you run the file main.tex through LaTeX.
\chapter{Metode Penelitian}
% Metode penelitian dibuat mencakup bahan dan instrumen yang dibutuhkan, deksripsi subjek
% penelitian, cara pengambilan contoh/ sampel, cara pengumpulan data, dan gambaran jelas pengolahan
% dan analisis data yang akan dilakukan. Metode penelitian dipaparkan dengan kerangka sistematis
% untuk menjawab rumusan masalah dan mencapai tujuan dari penelitian.

Metode penelitian ini digolongkan menjadi dua jenis berdasarkan luarannya, yaitu metode prediksi dan web-server.
Penjelasan rinci tentang keduanya adalah sebagai berikut.

\section{Metode prediksi}
Pada penelitian ini, kami berkonsentrasi pada metode prediksi berbasis kernel yang telah terbukti sukses~\cite{Scholkopf:2001,2546}, 
khususnya untuk data masukan yang dapat dideskripsikan dengan banyak sekali fitur, sebagaimana data senyawa, protein dan jejaring farmakologi.
Pada metode berbasis kernel, data masukan (observasi) dipetakan secara implisit ke sebuah ruang fitur melalui 
fungsi mapping $\Phi: \mathcal{X} \mapsto \mathcal{H}$, di mana $\mathcal{H}$ adalah sebuah reproducing kernel hilbert space (RKHS) dengan kernel yang mereproduksi $ K: \mathcal{X} \times \mathcal{X} \mapsto \mathbb{R}$.
Sebuah fungsi $K(x,x')$ adalah sebuah fungsi kernel yang valid jika dan hanya jika untuk sembarang himpunan berhingga, fungsi itu menghasilkan matrik Gram yang simetris dan positive-definite. 

\subsection{Kernel gabungan (composite kernel)} 
Kami mengusulkan sebuah kernel gabungan yang mengintegrasikan informasi kesamaan dari ruang kimia, genomik dan farmakologi.
Hal tersebut karena kita yakin dapat merumuskan beberapa kernel independen dari tiga ruang tersebut.
Di ruang kimia terdapat ukuran kesamaan dari dua buah senyawa bioaktif~\cite{Ralaivola20051093}, 
sedangkan di ruang genomik diperoleh kesamaan dari dua buah protein~\cite{Borgwardt:2005}.
Sementara itu, di ruang farmakologi dapat diperoleh kesamaan antara jejaring farmakologi yang terbentuk dari interaksi senyawa (obat) dan protein.

Berdasarkan karakteristik matematik dari matrik Gram yang positive-definite, dapat diturunkan sebuah preposisi sebagai berikut~\cite{Cristianini:1999}.
Misal $K_1$ dan $K_2$ adalah kernel-kernel untuk $\mathcal{X} \times \mathcal{X}$, $\mathcal{X} \subseteq \mathbb{R}^n$, $a \in \mathbb{R}^+$, $0 \le \lambda \le 1$, $f(\cdot)$ adalah sebuah fungsi bernilai real di $\mathcal{X}$, $\phi: \boldsymbol{\mathcal{X}} \mapsto \mathbb{R}^m$, dan $\boldsymbol{K}$ adalah sebuah matrik $n \times n$ yang simetris dan positive semi-definite.
Maka fungsi-fungsi berikut adalah kernel yang valid:
$K(x,x') = K_1(x,x') K_2(x,x')$, $K(x,x') = aK_1(x,x')$ dan 
$K(x,x') = \lambda K_1(x,x') + (1-\lambda)K_2(x,x')$.
Terlihat bahwa kombinasi konvek dari kernel menghasilkan kernel yang valid. 
Menurut~\cite{Joachims01compositekernels}, kombinasi dua kernel yang secara terpisah berkinerja baik akan juga berkinerja baik sepanjang kernel-kernel tersebut independen (tidak mengekstrak fitur yang sama).
% Hal ini hampir mirip dengan prinsip kerja metode ensemble boosting yang menggabungkan hipotesis-hipotesis yang independen untuk menghasilkan hipotesis yang lebih baik.

\subsection{Metode prediksi berbasis kernel} 
Mula-mula, kami akan menggunakan metode kernel yang telah teruji, ``mature'', seperti support vector machine (SVM)~\cite{Scholkopf:2001}.
Di penelitian ini, SVM yang termasuk metode supervised-learning menghasilkan sebuah fungsi pemetaan $f: X \mapsto \{0,1\}$, 
di mana $x \in X$ adalah sebuah sampel data yang mengandung informasi tentang sebuah senyawa, sebuah protein dan jejaring farmakologi antar keduanya, 
sedangkan $y \in \{0,1\}$ mengindikasikan ada tidaknya interaksi antara senyawa dan protein tersebut.
Formulasi prediksi ini dapat juga dimodelkan sebagai supervised bipartite graph learning.

Dalam tahap mendesain prediktor, kami menggunakan data set publik yang disediakan oleh Yamanishi~\cite{BleakleyY09}.
Untuk melakukan evaluasi, kami mengikuti penelitian sebelumnya, yaitu dengan menghitung nilai area under curve (AUC) pada kurva ROC.
Selain itu, juga dihitung metrik-metrik performa standar dari confusion matrix, seperti akurasi, recall, precision.
Beberapa jamu (tanaman-tanaman) yang berpotensi tinggi juga akan diuji dengan metode docking dan secara kimia (wet-lab tests).

\section{Web-server}
\subsection{Antarmuka pengguna (front-end)}
Antarmuka web akan dibuat sedemikian rupa sehingga bersifat minimalis dan sederhana untuk mendukung fungsi utamanya yaitu 
memproses dan menampilkan prediksi interaksi antara senyawa bioaktif dan protein.
Pada halaman pertama (homepage) terdapat visualisasi bipartite graph dengan empat set (himpunan), yaitu (dari kiri ke kanan): tanaman (jamu), senyawa bioaktif, protein dan penyakit, serta sebuah tombol ``Search'' untuk memulai proses prediksi atau pencarian jika informasi interaksi sudah ada.
% Ilustrasi homepage ditampilkan pada gambar~\ref{fig:homepage}.

Setelah proses prediksi berakhir, halaman luaran akan muncul yang menampilkan gambar bipartite graph serupa dengan yang di homepage, 
tetapi dengan hasil prediksi interaksi (konektifitas) antara simpul-simpul dari himpunan senyawa dan protein.
Hubungan antara simpul-simpul pada himpunan tanaman dan senyawa didapatkan dari database terkait, misal 
KNApSAcK\footnote{http://kanaya.naist.jp/KNApSAcK/} dan Dr. Duke's Phytochemical and Ethnobotanical Databases\footnote{https://phytochem.nal.usda.gov/phytochem/search}. 
Begitu pula dengan hubungan antara simpul-simpul pada set protein dan penyakit, misal dari Uniprot\footnote{http://www.uniprot.org/}.

Evaluasi antarmuka pengguna dilakukan melalui uji publik atau survei.
Peserta survei adalah peneliti terkait dengan beragam latar-belakang pendidikan, misal biologi, kimia, ilmu komputer (bioinformatik), dan farmasi.
Selain itu, ada peserta yang tergolong educated-user, di mana akan diberikan beberapa tugas untuk diselesaikan dengan memanfaatkan web-server tersebut. \\
Saran dan keluhan dari peserta evaluasi akan digunakan sebagai dasar perbaikan sistem.

\subsection{Kode proses inti (back-end)}
Hasil prediksi antara simpul-simpul senyawa dan protein berupa garis koneksi (edge) yang berbobot (weights).
Bobot tersebut didapatkan melalui voting berbobot (weighted voting), yaitu sebuah bobot $w_e= \frac{1}{n} \sum_{i}^{n} \hat{e_i}$, dimana
$n$ adalah jumlah metode prediksi yang digunakan dan $\hat{e_i} \in \{0,1\}$ adalah hasil prediksi sebuah edge~$e$ oleh prediktor ke-$i$.
Kami berpendapat bahwa beberapa metode prediksi mempunyai kinerja yang berimbang, sebagaimana ditunjukkan oleh~\cite{Chen17082015}.
Oleh karena itu, beberapa metode prediksi akan digunakan secara bersamaan sehingga didapatkan nilai bobot untuk tiap koneksi yang dapat diterjemahkan sebagai skor kepercayaan hasil prediksi.

Hal lain yang kami usulkan adalah sebuah database yang komprehensif untuk keperluan prediksi interaksi senyawa dan protein.
Pada dasarnya, database tersebut merupakan database mirror (kopian) dari beberapa database yang terkait dengan tingkat kedalaman data tertentu;
data yang kami anggap detil atau jarang digunakan tidak dikopi, tetapi dirujuk ke database asal.
Terdapat juga fitur untuk memperbaharui database mirror secara otomatis.
Kami akan mengembangkan mesin penjelajah (crawler) database untuk mencari dan mengunduh data di database terkait.

% doxygen for documentation
Semua kode yang membentuk web-server kami ini, baik front- maupun back-end, akan dipublikasikan ke khalayak umum, sebagai kode open-source.
Dengan begitu, utilitas dari kode akan dapat ditingkatkan, terlebih dengan dukungan dari dokumentasi (manual) yang lengkap.
Untuk itu, dokumentasi kode akan dilakukan dengan Doxygen.%\footnote{http://www.stack.nl/~dimitri/doxygen/}.

% Hipotesis yang diajukan dalam penelitian ini adalah sebagai berikut:
% \begin{enumerate} [topsep=-10mm]
% \itemsep0mm
% \item 
% Kernel, sebagai ukuran kesamaan dua entitas, merupakan metrik yang paling sesuai 
% untuk mengintegrasikan informasi dari ruang kimia, genomik dan farmakologi

% \item
% Metode prediksi berbasis kernel dapat memberikan akurasi yang tinggi, di samping membutuhkan sumber daya komputasi yang rendah dan tidak bergantung pada ukuran informasi untuk menyimpulkan kesamaan dua sampel

% \item 
% Voting berbobot dari beberapa metode prediksi yang digunakan mengindikasikan skor kepercayaan terhadap hasil prediksi 

% \item 
% Web-server yang menyediakan layanan prediksi memiliki daya-guna tinggi melalui desain antarmuka yang user-friendly, visualisasi yang informatif dan interaksi yang responsif,
% serta kode yang open-source dan terdokumentasi dengan baik
% \end{enumerate}