%% This is an example first chapter.  You should put chapter/appendix that you
%% write into a separate file, and add a line \include{yourfilename} to
%% main.tex, where `yourfilename.tex' is the name of the chapter/appendix file.
%% You can process specific files by typing their names in at the 
%% \files=
%% prompt when you run the file main.tex through LaTeX.
\chapter{Pendahuluan}
% Pendahuluan mencakup :
%  Latar belakang topik penelitian, didukung oleh perkembangan penelitian serupa yang telah
% ada sebelumnya
%  Posisi penting dan hasil yang diharapkan dari penelitian
%  Kontribusi aplikatif yang diharapkan bagi ilmu pengetahuan/ dunia praktis

\section{Latar belakang}
Penderita penyakit diabetes di Indonesia semakin meningkat dari tahun ke tahun, 
yaitu 7.2~juta (2011), 8.5~juta (2013) dan 9.1~juta (2015).
Oleh karena itu, dibutuhkan usaha-usaha pencegahan maupun penanganan diabetes.

Keanekaragaman hayati nusantara merupakan sumber daya luar biasa untuk mendukung usaha tersebut.
Ada dua cara untuk memanfaatkan keanekaragaman hayati nusantara, yaitu:
\begin{itemize} [topsep=0mm]
\itemsep0mm
\item mengarahkannya menjadi suplemen makanan sehat, sebagai usaha preventif
\item menjadikannya obat herbal tradisional dengan kaidah pengobatan modern, sebagai usaha pengobatan.
\end{itemize}
Pengembangan keduanya memerlukan informasi tentang interaksi \emph{antara} senyawa bioaktif yang terkandung dalam tanaman \emph{dengan} protein yang merupakan biomarker untuk suatu penyakit, misal diabetes.

Proses penemuan obat konvensional memerlukan waktu dan biaya yang relatif besar.
Hal tersebut karena penentuan tingkat interaksi antara senyawa dalam obat dan protein terkait penyakit melibatkan banyak sekali kandidat obat.
Setiap kandidat memerlukan uji klinis yang mahal dan lama.
Dengan demikian, untuk menghemat biaya, tenaga dan waktu, dibutuhkan \emph{prediksi} terhadap interaksi tersebut yang kemudian bertindak sebagai penapis massal \emph{in-silico} untuk menyeleksi kandidat senyawa obat baik alami ataupun sintetik.

Lebih lanjut, saat ini telah tersedia beberapa database publik yang berisi informasi-informasi krusial dalam menentukan tingkat interaksi senyawa-protein.
Beberapa database yang relevan antara lain, KNApSAcK, PubChem, BioAssay, dan ChemmineTools.
Informasi dari database tersebut dapat membantu merumuskan model prediksi interaksi yang akurat.

\section{Posisi penting dan hasil yang diharapkan}
Penelitian ini berkontribusi dalam proses penemuan obat baik herbal maupun sintetik, khususnya untuk penyakit diabetes.
Kesuksesannya akan mengurangi sumber daya yang diperlukan untuk menemukan suatu obat karena hanya kandidat yang berpotensi tinggi yang diuji secara klinis; dengan kata lain, ruang pencarian kandidat obat dipersempit.
Di samping itu, manifestasi hasil penelitian yang berupa aplikasi-web dapat digunakan oleh peneliti obat ataupun masyarakat umum.
Mereka dapat memperoleh informasi interaksi senyawa-protein, bahkan dengan masukan jenis tanaman dan penyakit.

Hasil yang diharapkan dari penelitian ini dapat digolongkan menjadi dua, yaitu: metode prediksi dan web-server.
\begin{enumerate} [topsep=0mm]
\itemsep0mm
\item Metode prediksi \\
Prediksi yang dimaksud adalah estimasi interaksi antara senyawa dan protein.
Penelitian ini mengajukan metode prediksi menggunakan kernel gabungan (composite kernel).
Kernel tersebut dirumuskan berdasarkan pengetahuan tentang tiga ruang sekaligus, 
yaitu ruang kimia, genomik dan farmakologi.
Selain itu, akan dikembangkan metode prediksi berbasis kernel yang akurat.

\item Web-server \\
Web-server sebagai manifestasi penelitian ini memiliki, paling tidak, dua kebaruan yaitu:
\begin{itemize} [topsep=0mm]
\itemsep0mm
\item data masukan berupa nama tanaman dan nama penyakit (sebagai tambahan untuk nama senyawa dan protein) dan 
\item skor kepercayaan terhadap hasil prediksi melalui voting beberapa metode prediksi
\end{itemize}
Web-server sejenis, misal DINIES~\cite{YamanishiKMSKG14}, SuperPred~\cite{NickelGEBRGDP14} dan SwissTargetPrediction~\cite{GfellerGWDMZ14}, berfokus pada obat sintentik sehingga wajar jika masukannya berupa nama senyawa dan protein.
Selain itu, prediksi yang dilakukan hanya menggunakan sebuah metode sehingga tidak bisa memberikan gambaran tentang skor kepercayaan prediksi.
\end{enumerate}

\section{Kontribusi aplikatif}
Kontribusi aplikatif dari penelitian ini meliputi:

\begin{enumerate} [topsep=0mm]
\itemsep0mm
\item
Prediksi yang akurat terhadap interaksi senyawa bioaktif dengan protein merupakan informasi yang krusial untuk pengembangan obat dan suplemen makanan, baik herbal maupun sintetis.
Dengan informasi ini sebagai penapis in-silico, uji klinis yang mahal dan lama hanya dilakukan pada kandidat senyawa yang berpotensi tinggi.

\item
Web-server publik yang handal berfungsi sebagai pusat informasi tentang (prediksi) khasiat tanaman obat.
Selain itu, kode web-server akan disediakan dalam bentuk open-source library dengan dokumentasi yang rapi dan lengkap sehingga dapat dimanfaatkan oleh peneliti.
Hal ini akan mendukung kesinambungan pengembangan dan perawatan sistem.
Kode tersebut meliputi fungsi-fungsi perhitungan kernel gabungan, proses prediksi, serta penjelajahan (crawling) database terkait.
\end{enumerate}



