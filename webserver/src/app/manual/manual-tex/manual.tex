\textbf{This website is free and open to all users and there is no login requirement.}

\noindent
\textbf{Website address:} http://ijah.apps.cs.ipb.ac.id \\
\textbf{Website name:} Ijah Webserver \\
\textbf{Authors:}
Vektor Dewanto (vektor.dewanto@gmail.com),
Wisnu Ananta Kusuma (w.ananta.kusuma@gmail.com),
Ajmal Kurnia (ajmalkuria@apps.ipb.ac.id),
Haekal Zidni Barkan (hzbarkan@gmail.com) \\
\textbf{Author Affiliation:}
Computer Science Department, Bogor Agricultural University, Indonesia

\noindent
\textbf{Input Data:} \\
The Ijah webserver takes as input,
plants (or compounds) in the drug-side and disease (or proteins) in the target-side.
It accomodates 3 use cases as follows.\\
\textbf{Use-case 1:}
having plants (or compounds) and diseases (or proteins),
users want to know 3 things:
a)~which compounds are contained in those plants (or which plants contain those compounds),
b)~which proteins are associated with those disease (or which diseases are associated with such proteins), and
c)~the interaction among those compounds and proteins, for which Ijah gives both factual and predicted results.\\
\textbf{Use-case 2:}
having only plants (or compounds),
users want to know:
a)~which compounds are contained in those plants (or which plants contains those compounds) and
b)~top-$n$ proteins based on the degree of interactions with those compounds, along with diseases associated with those proteins.\\
\textbf{Use-case 3:}
having only diseases (or proteins),
users want to know:
a)~which proteins are associated with those diseases (or which diseases are associated with such proteins), and
b)~\mbox{top-$n$} compounds based on the degree of interactions with those proteins, along with plants containing those compounds.
The Ijah webserver also accomodates command-line and programming interface in order to allow
programmatic access for rapid and efficient data retrieval.

\noindent
\textbf{Output:} \\
The Ijah webserver outputs interactions from plants to compounds to proteins to diseases.
Those interactions are visualized using a multipartite graph and formatted texts.
The thickness of an edge indicates the confidence level that the interaction between two vertices exists in the graph.
The Ijah webserver also provides downloads for most parts of its database.

\noindent
\textbf{Processing method:}\\
The database of Ijah webserver is essentially a partial mirror of
several major and related databases, such as KnapSack~\cite{pmid23292603}, DrugBank~\cite{pmid21059682}, and Uniprot~\cite{pmid25348405}.
We built database crawlers to autonomously extract essential information from relevant databases.
These crawlers are parts of our programming interface, which are publicly available.

The Ijah web-server estimates unknown interaction between a (natural) compound and a protein.
To this end, it has two strategies, namely:
a)~running its own predictors, i.e. an extension of BLM-NII~\cite{mei} and
b)~querying drug-target prediction servers, e.g. DINIES~\cite{YamanishiKMSKG14}, Superpred~\cite{NickelGEBRGDP14} and SwissTargetPrediction~\cite{GfellerGWDMZ14}.
Prediction results from several distinct predictors are then combined via
majority voting that determines the confidence level that interaction exists.
The prediction results are then stored in the database so that the same query later
will be as simple as table look-up.

\noindent
\textbf{Keyword:}\\
 drug-target prediction, herbal medicine, webserver, In-silico screening for drug discovery

{\footnotesize
\bibliographystyle{plain}
\bibliography{drugtarget_pred,predicting_webserver,searching_webserver,database}}
